\documentclass[bachelor]{iisthesis}
%           or master (bachelor or master is required)

% These two packages are highly recommended:
\usepackage[T1]{fontenc} % make non-ASCII characters cut&pastable in PDF
\usepackage{lmodern}     % easiest way to get outline fonts with T1 encoding

% \usepackage[ngerman]{babel}     % if the thesis is written in German
\usepackage{listings}
\usepackage{caption}
\usepackage{amssymb}
\usepackage{amsmath}
\usepackage{tabularx}
% allow multiline footnote
\usepackage[hang]{footmisc}
\setlength\footnotemargin{10pt}

\title{Robot Simulation and Motion Planning}
\author{Martin Griesser}
\supervisor{Emre Ugur, PhD\\Univ.-Prof. Dr. Justus Piater, PhD}

\definecolor{myblue}{rgb}{0.0,0.0,0.5}
\definecolor{mygreen}{rgb}{0,0.5,0.0}
\definecolor{myorange}{rgb}{1.0,0.4,0}

% code format settings
\lstset{
basicstyle=\footnotesize\sffamily\color{black},
commentstyle=\color{mygreen}\itshape,
frame=single,
emph={uint},
emphstyle=\color{myblue},
keywordstyle=\color{myblue}\bfseries,
showspaces=false,
showstringspaces=false,
stringstyle=\color{myorange},
tabsize=2
}

\lstdefinestyle{customxml}{
language=XML
}

\lstdefinestyle{customc}{
basicstyle=\footnotesize\sffamily\color{black},
commentstyle=\color{mygreen}\itshape,
emph={uint},
emphstyle=\color{myblue},
numbers=left,
numbersep=5pt,
numberstyle=\tiny\color{myblue},
keywordstyle=\color{myblue}\bfseries,
showspaces=false,
showstringspaces=false,
stringstyle=\color{myorange},
tabsize=2,
language=C++
}

\begin{document}
\maketitle

\chapternn{Abstract}
The first part of this thesis describes the design and implementation of a custom simulation solution for the robot setup in the IIS Lab. It describes the necessary modelling steps for the resulting simulation models and the implementation of a proper control interface based on the Robot Operating System (ROS). The resulting solution provides the same control interface as the real robot and can therefore be used to test high-level control code on the simulator before utilizing it on the real robot. Therefore it was necessary to determine the volumetric, kinematic and dynamic properties of the involved robot components and their exact placement relative to each other. The second part describes the integration of the motion planning framework MoveIt into the existing setup. The aim of this framework is to plan and execute complex robot motions without colliding with the environment. The proper functioning of the planning tools and the simulator is then shown on the example of a benchmark pick and place task that is planned and executed on the simulator and the real robot as well.


\chapternn{Acknowledgements}
First I want to thank my supervisor Emre Ugur who always had an open ear for my questions and greatly supported this project with his ideas. I also want to thank Alex Rietzler for providing technical assistance and giving useful feedback. 

Simon Hangl (Kukie Schnittstelle)
Simon Haller (Technische Hilfestellung, System, Git)
All other members of the IIS team that frequently provided very useful feedback.

\tableofcontents
\listoffigures
\listoftables
%\listoflistings

\uibkdeclaration
\label{chap:declare}

% Introduction
%!TEX root = thesis.tex

\chapter{Introduction}
\label{chap:introduction}

Robots play an increasingly important role in today's world because they enlighten human work in many different areas. Robots allow the automation of production processes but they are also utilized in other areas like medical surgery or even as household robots (autonomous vacuum cleaner, lawn mower, \ldots). Depending on the application domain, those robots require the ability to act in differing levels of autonomy. \citep{lavalle2006} describes the branch of \emph{robotics} as the area of automating mechanical systems that have sensing, actuation and computation capabilities. Part of the research in that branch is to create control software that allows to autonomously perform high-level tasks like grasping and object manipulation on robot hardware. The design and implementation of such high-level control software for a robot is a cumbersome task. The algorithms need to be tested and debugged during implementation process, but those tests come with a high level of risk. Incorrect algorithms can lead to damages on robot components or their environment and entail costly repairs. In the worst case even people can get hurt by uncontrolled robot motions.\\

The solution to those problems is the usage of a simulator that mimics the robot and its behaviour as accurate as possible. Therefore it has to provide the same control interface to allow to test and debug each part of the software on the simulator before utilizing it on the real robot. Using a simulator allows to easily evaluate different design approaches and algorithms during the software development process. It can act as replacement for the real robot and facilitates parallelization of testing and debugging tasks, as the robot may be blocked by other persons or unavailable at certain times. It also allows to skip the technical overhead that often comes with working on the real device. Those considerations motivated the simulation part of the thesis. \\

The second part of the project focuses on robot motion planning. An autonomously acting robot needs to be able to plan its motions. Moving a robot's hand for example cannot follow any arbitrary path towards a target pose. During that motion it might collide with itself or any other obstacle within it's environment. That means those paths have to be planned carefully to avoid accidental collisions and to generate smooth and well controlled motions with respect to the limits of the robot. The resulting trajectories should be preferably short and may not contain unnecessary motions. The project targets are specified in the following section.

% Without motion planning ability those motions have to be constantly specified by a human operator which is only suitable for repetitive tasks.
%Robots are utilized for automation of production processes, medical surgery,... Usually they are programmable, multi functional mechanical systems.Industrial robots utilized in factories that perform repetitive tasks, high precision medical robots, household robots (automatic vacuum cleaner or lawn mower). Robots can be arbitrary intelligent - research tends to create robots that are able to act largely autonomous. Research in the branch of \emph{robotics}. \citep{lavalle2006} describes the term \emph{robotics} as the area of automating mechanical systems that have sensing, actuation and computation capabilities. and research tends to create control software that is able to perform high-level tasks like pick and place actions on robot hardware (translate into low-level control commands, executable on the hardware). An important property of an autonomous system is it's ability to plan its motions. It should be able to move towards a target position without colliding with obstacles in it's environment. Moving the robot hand cannot follow any arbitrary trajectory towards a target pose. During that motion it might collide with itself or any other obstacle within it's environment. That means those trajectories have to be planned carefully to avoid accidental collisions and to generate smooth and well controlled robot motions. The planned paths should be preferably short and may not contain unnecessary motions. Without motion planning the motions have to be constantly specified by a human operator - not preferable, only suitable for repetitive tasks. \\


%Therefore it needs to be able to move towards target positions without colliding with obstacles in its environment. Research in the branch of robotics tends to enable robots to perform certain tasks largely autonomous. Without motion planning the motions have to be constantly specified by a human operator - not preferable, only suitable for repetitive tasks.

%The second part is about robot motion planning. Moving the robot hand cannot follow any arbitrary trajectory towards a target pose. During that motion it might collide with itself or any other obstacle within it's environment. That means those trajectories have to be planned carefully to avoid accidental collisions and to generate smooth and well controlled robot motions. This also involves to create and maintain an internal representation of the robot and it's environment, a step that is common to the simulation part of the thesis. Chapter \ref{chap:moveit} shows the configuration and integration of the motion planning framework MoveIt into the IIS-Lab robot setup along with some usage examples.

\section{Project Targets}

The first goal of the project is to create a realistic replication of the IIS-Lab robot setup, using a suitable simulation platform. The solution needs to be able to generate proper sensor data and provide the same ROS control interface as the real robot. Additionally it would be preferable that the solution is able to detect and visualize accidental collisions of robot components with the environment. The necessary steps are explained in Chapter \ref{chap:simulation}.\\

The second objective is to choose and integrate a state of the art motion planning framework into the existing setup. The required functionality includes solving inverse kinematics\footnote{Problem of finding possible joint settings for the robot to achieve a desired end effector position and orientation in Cartesian space \citep{craig2005}} (IK) problems and planning collision-free trajectories for complex robot motions in joint space and Cartesian space. This integration process is described in Chapter \ref{chap:moveit}. The proper functioning will then be shown by planning and executing a benchmark pick and place task on the simulator and the real robot as well. Chapter \ref{chap:pick_place} shows how planning problems are solved and the resulting motion plans are executed on the robot.\\

The implementation of those objectives requires to determine the kinematic, dynamic and volumetric properties of the involved robot components. This includes to do an exact measuring of the robot setup and determine the placement of the involved components relative to each other to be able to create an internal representation of the world. 

%For Cartesian positioning functionality it is also necessary to define a world reference frame and place components relative to that reference frame. \citep{craig2005} describes the inverse kinematics (IK) problem as the problem of finding possible joint settings for the robot to achieve a desired end effector position and orientation in Cartesian space. Forward kinematics (FK) is the reverse problem of finding the position and orientation of the end effector, given a set of joint angles. The end effector is the tool, mounted on the tip of the robot arm.

\section{The IIS-Lab Robot setup}
\begin{figure}[ht]
	\centering
  \includegraphics[width=0.75\textwidth]{images/robot_setup.jpg}
	\caption{Current setup in the IIS-Lab}
	\label{fig:iis_setup}
\end{figure}

The structure of the robot setup in the IIS-lab changes frequently, as new robot components are introduced and arrangements are modified. The setting considered within this thesis is a snapshot though the solution that needs to be developed should be customizable to reflect alternating settings. The main part of the robot setup consists of an aluminium torso with two mounted 7 DOF\footnote{Degrees of freedom - the number of independent variables, necessary to describe a robot's configuration} KUKA LWR4+ industrial robot arms, as can be seen in Figure \ref{fig:iis_setup}. The IIS-Lab also owns two Schunk SDH 3 finger grippers that can be mounted to the robot arms for grasping experiments. Additionally there is a Kinect camera placed on the torso, between both arms, that provides RGB and depth images of the task environment. Control and the data exchange with those components is based on ROS which will be described in the next section. The workspace where experiments usually take place consists of a table in front of the robot which is covered with a foam mat for security reasons. 

\section{The Robot Operating System (ROS)}

The implementation of the project requirements is based on ROS. Therefore a brief introduction about the basic concepts\footnote{http://wiki.ros.org/ROS/Concepts} shall be given here. The explained terminology will be used throughout this thesis. As stated in \cite{quigley2009}, ROS is not an operating system in the classical sense. It runs on top of a host operating system (usually linux) and can be seen as an additional communication layer, providing various mechanisms for inter process communication. A ROS system consists of a number of \emph{nodes}. Each node is an independent computation unit that runs in it's own process, adding some clearly defined functionality to the overall system. For example one node can be responsible for planning, another one for perception and a third one for controlling the hardware. Nodes communicate to each other by passing \emph{messages}, using the ROS communication infrastructure. Messages are strictly typed data structures, defined in a special message composition format\footnote{http://wiki.ros.org/msg}. They can be composed of primitive types like float, integer or string, but also of other message types. Therefore it is possible to create arbitrary complex messages for each use case. Messages are published to \emph{topics}. A topic is a strongly typed message bus, addressed by it's \emph{topic name}. Arbitrary nodes can connect to a topic in parallel, as long as they use the correct message type. Each node can publish and subscribe to a number of topics. It is also possible that various nodes publish to the same topic. 

Topic names are strings, used to identify topics. They can be organized into \emph{namespaces} to build a tree hierarchy comparable to the directory structure in a file system. This is very important, as for example the simulator should use similar topic names as the real robot. The namespace concept allows both instances to use identical names but each one in it's own namespace. The following samples represent valid topic names:
\begin{itemize}
\item \texttt{/} (this is the root namespace)
\item \texttt{/topic}
\item \texttt{/component/topic}
\item \texttt{/namespace/component/topic}
\end{itemize}

The communication via ROS topics is asynchronous - involved nodes may even not be aware of each others existence. Synchronous message exchange between nodes happens via ROS \emph{services}. In contrast to topics, a service with a given name can only be offered by one single node. Services are addressed, using the same naming strategy as topics. The service message is composed of a request and a response part. A client node that sends a service request will block, until the advertising node has handled the request and delivers a response. The concept of ROS topics and services is shown in Figure \ref{fig:ros_concept}
\begin{figure}[h]
	\centering
  \includegraphics[width=0.5\textwidth]{images/ros_concept.jpg}
	\caption{ROS nodes, topics and services}
	\label{fig:ros_concept}
\end{figure}

The nodes of a ROS system can be distributed over various different machines. One of them has to be the dedicated \emph{ROS master}. The master is responsible to handle topic and service registrations and holds information about the involved ROS nodes. Other machines connect to the master via network. The ROS master also provides a centralized \emph{parameter server}. This is a shared dictionary that can be used to store and retrieve configuration data and other shared parameters. Nodes can access the parameter server at runtime and read or modify it's content. \\

A system usually consists of a large number of nodes that have to be configured and started. This can be done, using so called \emph{launch files}. Those are simple textfiles, holding startup information and configuration details for one or more nodes in an XML like syntax. Using the \emph{roslaunch} command line tool, a whole system of nodes can be configured and launched at once. \\

ROS is a modular software system organized into \emph{packages}. Each package adds clearly defined functionality and can be reused in other systems. Custom functionality is added to a ROS system by creating a new package and developing the required piece of software. A package might contain one or more ROS nodes or even only configuration data. Existing packages are usually installed, using a software repository package manager. A very useful ROS package is the visualization tool \emph{RViz}. This tool provides a number of different plugins that allow to display robot setups and configurations, planned trajectories or point cloud data from a vision sensor. RViz gets utilized in the motion planning related part of this project.
% Simulation
%!TEX root = thesis.tex

\chapter{Robot simulation}
\label{chap:simulation}

This chapter focuses on the simulation part of the thesis. The objective is the creation of a simulation model, particularly designed for the IIS lab robot setup. The model has to reflect the properties and behaviour of the contained robot components as good as possible. Certainly the simulated components have to provide the same control interface as their real counterparts, allowing to test and optimize control code on the simulator before utilizing it on the real robot. Preferably, the control code sees no difference about on which instance it is executed. The recommendations on such a solution can be summarized as follows: 

\begin{itemize}

\item
The simulator needs to be able to generate realistic sensor and feedback data that can be used by the control software. This includes forces and torques that are measured within force sensors, the current state of the various robot joints (position, velocity, effort), but also RGB and depth images, usually produced by Kinect cameras and vision sensors.

\item
The simulation solution has to provide exactly the same ROS control interface as the real robot. This interface essentially consists of a number of ROS topics, that can be used to send control commands to the various different robot components or to read actual joint states and sensor data.
 
\item
The utilized simulation platform has to provide a graphical user interface that allows to visualize the motions of the robot and it's interaction with the environment. Possibly accidental collisions of  robot parts have to be registered and should also be visualized.

\item
In order to be used by as many people as possible, it is very important that the solution is really easy to use and does not require a long lead time. Therefore it has to be focused particularly on usability.

\end{itemize}

The following sections explain in detail, how this goal was reached. At the beginning stands the process of finding a suitable simulation platform that meets the requirements and the considered criteria. After that, the chosen simulation platform V-Rep is introduced and an overview about how to design dynamic simulations is given, explaining some of the necessary terminology. Subsequent sections focus on the necessary steps to achieve the final solution, namely finding and modelling required robot components, assembling and configuring the final simulation scene and the implementation of the ROS control interface.

\section{Choosing a suitable simulation platform}

The tasks executed on the robot are in most cases variations of so called 'pick and place' tasks. An object gets picked up, lifted and placed somewhere else within the robot's workspace. Therefore joint target positions are sent to the control interfaces of the various robot components and they execute the commanded motions if possible. The question, if the execution is possible at all and in that case in which velocity, is influenced by a number of physical parameters. The maximum effort of the motors in the joints is limited. If the force that acts upon a joint is higher than the maximum effort of the motor it will not be able to maintain its current position or to reach the desired target position. This can happen if the picked object is to heavy or if the robot collides with another immovable object in it's environment. The forces that act upon each single joint are influenced by a number of parameters like the position within the kinematic chain of the robot, the summed weight of the robot components themselves and also the weight of a possibly additional payload.

The required solution should be able to provide a realistic simulation of those dynamic interactions. Therefore the utilized simulation platform has to take use of a powerful physics engine. A physics engine is a software component, that is capable of computing parameters of physical processes and the dynamic properties of the involved objects. Examples for such engines are the Open Dynamics Engine\footnote{http://www.ode.org} (ODE) and Bullet physics\footnote{http://bulletphysics.org}. Some of the evaluated simulation platforms even provide a number of different physics engines to choose.\\

The candidates that have been taken into account were Gazebo\footnote{http://gazebosim.org/}, V-Rep\footnote{http://coppeliarobotics.com}, MORSE\footnote{http://www.openrobots.org/wiki/morse/} and Openrave\footnote{http://openrave.org}. After some investigation only two of them had been evaluated in greater detail. Criteria for the selection had been:

\begin{itemize}
\item
Which physics engine is used respectively is it possible to choose among various engines?
\item
Usability and stability
\item
Expandability
\item
Availability of required model components (arm model, gripper,...)
\item
Quality of the documentation
\item
Licence issues
\end{itemize}

Taking into account those criteria it went clear that V-Rep will be the simulation platform of choice. In some initial tests V-Rep seemed to be much more stable than Gazebo and the user interface is very intuitive. Another important point is that V-Rep ships with a fully functional model of the KUKA LWR4+ robot arm.

\section{The Virtual Robot Experimentation Platform(V-Rep)}
V-Rep is a powerful robot simulation platform, developed by Coppelia Robotics. The current version (V3.1.2) provides the ability to choose from three configurable physics engines (ODE, Bullet, Vortex -- only trial version) for simulating dynamic processes. It also contains a very comfortable editor for modelling robot components and simulation scenes. In the shape-edit mode it is possible to edit and simplify meshes. This is very important because for simulating dynamic processes only simple shapes with a low amount of vertices, edges and faces should be used to reduce complexity. The contained model browser has a rich set of different robot models, static objects and various sensors, ready to use. An additional feature is V-Rep's configurable collision calculation module that can be used to detect and visualize all kinds of collisions (self-collisions, collisions with the environment). The behaviour of the simulator is highly customizable via a rich programming API for C++ as well as the scripting language LUA\footnote{http://www.lua.org}. V-Rep is no open source software but it provides a free licence for educational units and can therefore be used for research purposes. (TODO reference and citation of V-Rep paper!!!)

\section{Dynamic simulations in V-REP}
For a better understanding of the modelling process it is necessary to explain a few fundamental concepts about designing dynamic simulations in V-Rep. This section just covers those aspects that are important for the underlying project. A more detailed explanation can be found in the official V-Rep documentation\footnote{http://www.coppeliarobotics/helpFiles}. \\

Each simulation scene in V-Rep is composed from a number of models that are arranged within the environment. A model consists of various scene object, combined in a tree like structure to mimic the kinematic chain of a robot component. Each model has a dedicated model base and constitutes a sub-tree of the scene hierarchy. The model base is the root element of the model tree. There are existing various types of scene objects within V-Rep, but only those, which are important for the implementation will be explained here.

\begin{itemize}
\item \textbf{Shape} \\
A shape is a 3 dimensional body. Shapes represent the visual parts as well as the dynamically enabled parts of the scene. It is necessary to distinguish between primitive shapes (Cylinder, Cuboid, Sphere, Plane and Disk) and complex shapes (triangle meshes). Primitive shapes are much easier to handle for the physics engine as there can happen a lot of optimization during dynamics calculations. Complex shapes usually look better and therefore they are used mainly as the visual part of the model and by the collision detection module. Various shapes can be combined to groups and therefore treated as one single object. Shapes can be defined as static or non-static objects. The position of a static object is fixed relative to it's parent node within the scene hierarchy and will not change during simulation. Non-static objects underlie gravity and will fall down if they are not constrained by a dynamically enabled joint or a force sensor connection. It is also necessary to distinguish between respondable and non-respondable shapes. Respondable shapes have a clearly defined mass and moment of inertia and therefore they create collision reactions when colliding with other respondable objects during simulation. Only respondable shapes are considered during dynamics calculation. Usually a model in V-Rep is composed of a visual part, consisting of complex shapes and a hidden part, consisting of groups of primitive shapes that are configured for the dynamics calculation. This issue will be covered again when explaining the creation of the hand model.

\item \textbf{Joint} \\
A joint is a flexible connection between two rigid parts of a robot. It has to be distinguished between revolute or prismatic joints with one degree of freedom and spherical joints with three degrees of freedom. In the arm and hand model only revolute joints are used. Joints can be passive or actuated by a motor (explain possible configuration parameters!!!). V-Rep provides various different joint modes, but only the torque/force mode and the IK mode are used within this project. (Maybe explain the two modes in greater detail...)

\item \textbf{Vision sensor} \\
States a simulated image capturing device. A vision sensor can capture images of the simulation scene, depending on it's configuration. It can deliver RGB image sequences as well as depth images. Vision sensors are used in the Kinect camera model.

\item \textbf{Force sensor} \\
A force sensor in V-Rep is a rigid connection between two dynamically simulated objects. The calculated forces and torques can be measured and visualized. It is also possible to define a maximum force the connection is able to bear. When this maximum value is exceeded, the connection will break.

\item \textbf{Dummy} \\
Dummies are the simplest scene objects at all but they provide some important functionality, especially for the various calculation modules of V-Rep. They can be understood as the origin of a named reference frame with a configurable position and orientation within the planning scene. Two dummies can be linked as 'tip-target' pairs to be used by the IK calculation module, or to follow a predefined path within the scene. The importance becomes more clear when the construction of the simulation scene will be explained. 

\end{itemize}

Groups of scene objects can be organized in collections and treated as one single entity. Collections play an important role in the collision detection module.

Some words about calculation modules (collision detection module, IK calculation module).


\section{Designing the simulation scene}
This section explains in detail the structure of the simulation scene and the contained model components. Three types of robot models are used - the KUKA LWR3+ model, the Schunk SDH gripper model and the Kinect camera model. The arm and camera model are part of the V-Rep model browser but the gripper model had to be constructed from scratch.

\subsection{Modelling the Schunk SDH gripper}

The gripper has three fingers. One of them is fixed around the z-axis and cannot be rotated. Two fingers can be rotated 90° around the z-axis, but they are connected contrariwise. That means if one finger rotates to the left, the other one is rotated to the right for the same distance.
The modelling process started with the search for suitable meshes for the rigid parts of the gripper. They have been found in the schunk\_description ROS package from the schunk\_sdh stack. Those meshes have been arranged according to the technical description. After placing the meshes the 8 joints have been added at the correct locations. It was important to determine the correct location and orientation for each single joint within the model to allow the correct movement of the fingers. This was achieved by using the shape edit mode and select the circle shaped area within the mesh, where the joint had to fit. From that selection a cylinder shape was extracted and the joint was then centered within this newly created cylinder. Those steps had to be repeated for all 8 joints. The placement process can be seen in Fig??? After placing all the joints and links within the scene, the model tree was adjusted to form the kinematic chain of the hand. The dynamic parameters of the joints were set according to the technical description. Those parameters include the joint limits, maximum velocity and maximum effort. As the root joints of the 2 fingers are connected, the setting of the second finger root is configured sightly different. The joint is operated in the 'dependent mode' and the dependency equation just mirrors the position of the connected joint.

The meshes only form the visual part of the model as they are to complex to be used for dynamics calculations. So the shape of each link had to be approximated by groups of pure shapes. This has been achieved by executing the following steps for each single part of the model:

- within the shape edit mode extract parts of the mesh that could be approximated by a primitive
  shape (cuboid, cylinder), by selecting suitable groups of vertices
- extract the corresponding shape by using this V-Rep editor functionality
- repeat those steps until the most important parts of the link are approximated that way
- group those primitive shapes to treat them as one single object
- adjust the dynamic parameters (mass, material settings, inertial matrix)
- adjust the local respondable mask
- give the group the same name as the corresponding mesh, but with the '\_respondable' suffix
- remove the extracted shapes from the current visibility layer because they are just used for the
  dynamics calculations

The extraction process is visualized in Fig???
Dynamic parameters like mass and inertial matrix have been provided by Alex Rietzler. The predefined 'highFrictionMaterial' setting was used for each single part of the finger because that showed better results when picking up objects later on.

The last steps of the modelling process were the adjustment of the model hierarchy. The root element is he respondable part of the wrist which is also defined as the model base. It is important to follow the V-Rep guidelines for designing dynamic simulations because if the hierarchy is wrong, the model will simply fall apart when starting the simulation (see corresponding chapter in the documentation for more information). Each non-static and respondable shape has to be connected to it's parent by a joint or a force sensor. The visual part of the link is always a child object of the corresponding respondable. That way, the kinematic chain of the gripper is formed.(TODO: insert images!!!)


\subsection{Putting the pieces together}

After finishing the modelling process of the gripper each necessary component was available to build up the simulation scene. Alex Rietzler had already created a mesh for the robot torso which was ready to use. The two robot arms where taken from the model browser and inserted into the scene. A dummy was placed at the bottom of each arm which will be used as the origin of the arms private reference frame. The exact translation of each arm was also already available so it had just to be configured. The grippers were placed on the tip of each arm. The correct rotation and offset was measured on the real counterpart and then configured accordingly. 

The plate and the legs of the table are modelled as group of primitive cuboids in the correct size. The table was defined as respondable, which means it will produce a collision reaction if a robot component collides with it. But as it is a static object it will not show a reaction to such a collision because it's position is fixed within the scene. The chosen material setting is also the 'highFrictionMaterial' but that can easily be changed by modifying the dynamic properties of the table plate.

The kinect camera model was also taken from the model browser. As it's position and orientation in the real world is not fixed and might change from time to time, it's position within the simulation scene is only an approximation to reflect the real world setting as good as possible. In the vision sensor settings, the option 'Ignore RGB info' was selected. A vision sensor usually produces two images, namely a color image and a depth image on each simulation pass. This generation process slows down the simulation and therefore skipping one of the two images brings a speedup.

The origin of the overall reference frame is located on the upper left side of the table, indicated by a slightly green shimmering sphere. The position and orientation of the torso, the table and the two arms was were configured relative to this reference frame. A dummy object called 'ref\_frame\_origin' was placed at that location. Each position calculation later on will happen relative to that dummy. If it is necessary to move the origin to another location within the workspace this can simply be achieved by just moving that dummy to the required location.



\subsection{Configuring the collision detection module}

The solution should be able to detect and visualize possible accidental collisions of one of the robot components with their environment. It would also be good to have some kind of warning if one of the robot components comes dangerously close to an obstacle during a movement. Therefore a so called 'collision shield' is modelled around the two robot arms, to detect if the arm comes closer than approximately 5cm to an obstacle. To achieve that functionality, for each single link another shape had to be modelled that is slightly larger than the corresponding link and that only purpose is to be checked for collisions. The modelling process started at the second link of each robot arm and on each link the following steps had to be performed:

- Copy the visible part of the robot link
- Morph the copied object into a group of convex shapes (V-Rep editor functionality)
- Ungroup the resulting shape and merge it into one single shape
- Grow the resulting shape in x and y direction of it's own reference frame to receive a mesh that
  is approximately 10cm larger than the original mesh but has the same height.
- Adjust the outside color to make it green and nearly transparent
- Rename the shape to the same name as the original shape, with the '\_col' suffix 
- Make the new shape a sibling to the original one within the model tree
- Define the new shape to be static and non-respondable.
- Disable the 'collidable' flag on the shape. This behaviour will be overridden in the collection
  settings later on when configuring the collision detection module

The modelling process is visualized in Fig???.

Basically there are two different types of possible collisions:

- Soft collision - the collision shield hits another collidable object 
  (no direct collision - only a collision warning)
- Hard collision - a robot component directly hits another collidable object. This would also
  be a collision in real world.

The solution should be able to distinguish clearly between those two types of collisions. Therefore a number of so called collision objects have to be configured within V-Rep's collision detection module. Each collision objects checks for collisions between a collider against a collidee. Collider and collidee are single shapes or collections of shapes with the collidable flag set. For each arm two collision objects have to be set up - one for each collision type. The configuration is explained for the left arm but the steps for the right arm are similar. The first collision object is named 'left\_arm'. The collider is the collection called 'leftArm'. This collection contains the whole sub tree of the left arm, including the gripper, but excluding the elements of the left collision shield. The collidee are all other collidable entities within the scene. This collision object detects the hard collisions. This also makes clear, why the collidable flag is disabled on the parts of the collision shield because otherwise collisions with the shield would also be detected which would not be correct. The second collision object is called `left\_armShield'. The collider is the collection called `leftArmShield'. This collection is composed of the parts of the left collision shield and the whole subtree of the left gripper. This collection is defined to override the value of the collidable flag of the contained elements - this is important because as previously mentioned, the elements of the collision shield are defined as not collidable. The collidee of the current collision object is the collection called `exLeftArmShield'. This collection simply contains all other scene objects except those contained in the left arm's subtree. As mentioned before, the same configuration steps happened for the right arm. All collision objects are defined not to be handled explicitly. This means that V-Rep does not check for collisions automatically in each simulation pass. It has to be done manually and will be explained later on in the section about the control interface implementation. 

\subsection{Configuring the IK calculation module}

The IK calculation module can be used to define different IK groups to solve the inverse kinematics for the required robot components. This functionality is used for the Cartesian positioning 
functionality of the ROS control interface. To define an IK group it is necessary to exactly define the kinematic chain of the robot component, starting at the base link up to the dedicated tip. The base link is the root element of the arm's model tree. The tip element is a dummy that is placed on the top end of the arm. That dummy must be a child element of the last link in the arm tree and it defines the reference frame that is used for Cartesian positioning. All the joints between the base and the tip, that are operated in IK mode will be taken into account by the IK calculation module. An additional dummy is used to define the target pose for IK calculation. Initially the pose of the target dummy is exactly the same as that from the tip dummy. When moving the target dummy to a new location, the IK calculation module tries to bring the tip dummy exactly to the same pose by setting the joint values of the manipulator accordingly, respecting the joint limits and the configured constraints and tolerances. More than one IK group can be defined for each manipulator. To improve performance three IK groups have been configured for each arm. The first one focuses on performance but the settings provide less stability. The second one is more or less the same configuration but with a little higher tolerance values. The third one is designed to increase stability especially for positions close to singularities. But that configuration is less performant and therefore it is only used if the other configurations have failed to find a solution. How those IK groups are used is explained in the ROS control interface section.

\section{Implementation of the ROS control interface}

\subsection{Overview}

After finishing the modelling process it was necessary to find a proper way to control the robot components via a ROS control interface. The arm as well as the hand is controlled via a set of inbound and outbound ROS topics that allow to send commands to the underlying component or to receive state data from the component(joint states, sensor data,\ldots). Each simulated component should provide exactly the same interface as it's real counterpart.

One problem is how each type of component can be clearly identified within the current simulation scene. A scene is a hierarchy of various scene objects, organized in a tree structure. As already explained, those objects can be shapes, joints, sensors or even only dummies. The scene content can be modified by the user. Maybe a gripper gets replaced by another component, an arm gets removed or an additional Kinect camera gets installed. The required solution should be able to react to changes in the current scene. Parts of the model hierarchy should be clearly identifiable as a specific simulation component. Each single part of a component should be identifiable (joints, sensors, dummies, IK groups, collision objects). Luckily V-Rep provides various extension points for programmers and is therefore highly customizable. 

After some investigation about the possibilities the decision was made to create a simulator plugin, using the V-Rep regular API. This approach states the most flexible solution as this API provides more than 400 functions. A plugin is a compiled library file, written in C++ that has to follow some V-Rep specific naming conventions and must reside in the V-Rep working directory. The library file gets automatically loaded on V-Rep startup and runs in the main simulation thread. This means that it has to be programmed really carefully to avoid performance leaks during simulation. The plugin has to provide a clearly defined interface, consisting of 3 functions:

- v\_RepStart - called on startup and can handle some initialization
- v\_RepEnd - called before shutdown and can do some cleanup
- v\_RepMessage - called very often during the whole V-Rep lifecycle and is therefore a very
  performance critical method. Via this function V-Rep notifies the plugins about events like
  start/end of simulation, simulation step, scene content change, scene switch, \ldots.
  The plugin code can react to those events accordingly.


\subsection{Plugin Architecture}
The plugin code is organized as can be seen in Fig???.

- SimulationComponent
  A simulation component is a single, reusable part of the robot that can be used in different
  environments. Each component provides it's own clearly defined control interface. A simulation
  scene can contain various types of components. The SimulationComponent class is the abstract
  base class for all simulation components. Currently there are existing two concrete implementations
  -- the LWRArmComponent and the SchunkHandComponent. If the scenario should be extended and new
  components are introduced it is necessary to create a subclass of SimulationComponent and provide
  implementations for the abstract methods. Each SimulationComponent consists of two parts. The first 
  one is a class that provides access to a concrete simulation component instance. The LWRArm class
  for example stands for a KUKA LWR4+ arm model in the scene. This class provides full access to the
  functionality of the underlying arm model.
  
  The second part is a controller class for the component instance. This controller encapsulates the
  whole ROS interface to the simulation component and has to be a subclass of the abstract 
  ComponentController class. On each simulation pass the controller publishes all the necessary
  state data to it's various topics and sends incoming commands to the simulation component. The
  names of the various provided topics are composed from the overall namespace, the defined unique
  name of the component and the actual topic name. For example the joint control topic of the right
  robot arm evaluates to `/simulation/right\_arm/joint\_control/move'
  
  On simulation start each SimulationComponent registers it's ComponentController at the ROSServer
  and unregisters it on simulation end.
  
- ComponentContainer
  This class represents the set of all identified simulation components in the current simulation
  scene. On V-Rep startup an instance of ComponentContainer is created. Each time, the content of
  the current simulation scene changes, the method `ComponentContainer::actualizeForSceneContent' 
  is triggered. This method performs the following steps:
  - It validates each currently registered SimulationComponent instance if it is still valid and
    present in the scene.
  - It traverses the whole scene hierarchy to identify newly created components
  - If a new component is identified, a corresponding concrete instance of SimulationComponent 
    is created and added to the container.
  To identify a component it has to be marked by using V-Rep's custom developer data functionality.
  Details are explained further on.
  
  The ComponentContainer gets notified about each simulation step. It then simply forwards that
  message to all registered components. Those can then perform all necessary steps like triggering
  collision checking or the IK calculation module.
  
- ROSServer
  The ROSServer is a static class that encapsulates all ROS related functionality. It tries to 
  initialize ROS on startup. If the connection to the master can be established it creates a
  ROS NodeHandle for the `simulation' namespace. Otherwise it forces the plugin to unload, because
  it is not able to work without a running roscore.
  Each SimulationComponent registers it's ComponentController instance at the ROSServer. On 
  simulation start it initializes the registered controllers with the maintained NodeHandle. The
  ROSServer gets also notified about each simulation step and forces the controllers to handle the
  received commands and publish all the necessary data.
  On simulation end it forces the registered controllers to shutdown their publishers and 
  subscribers.
  
- ComponentController
  This is the abstract base class for all controllers. A ComponentController gets initialized by
  the ROSServer on simulation start. Concrete implementations can use the provided NodeHandle to
  create all the necessary publishers and subscribers. The update method is called by the ROSServer
  on each simulation step and forces the controller to publish all the required data. On simulation
  end the shutdown method is called by the ROSServer, forcing the controller to shutdown all 
  publishers and subscribers.
  
- LWRArm
  This class provides access to a correctly configured LWR arm model within the simulation scene.
  To fullfill the requirements for the control interface, the arm has to be able to operate in joint
  control mode and in inverse kinematics mode. Initially the arm starts in joint control mode. All 
  the joints are switched to torque/force mode and accept target positions to be set. The simulated
  PID controllers will try to move the joints to their designated target positions. 
  
  Switching to IK mode means to operate all the joints in IK mode and use the previously configured   
  IK calculation module. Setting a target pose in Cartesian space means to bring the IK target dummy
  into the required pose. The IK calculation module tries then to bring the linked IK tip dummy into
  the same pose by commanding the robot arm's joints accordingly and satisfying the configured
  constraints and precision settings. At the moment, three IK groups are configured for
  each arm with different settings to achieve performant and stable solutions. Those groups are
  sequentially called until one of them is successful. The configuration of the first group focuses
  on performance but provides less stability. The last group uses a configuration that is slower but
  provides more stability in positions closed to singularities. If none of the groups was successful
  it will result in an error message on the console, otherwise the arm will start or continue to move
  towards it's target pose. When initializing the LWRArmComponent, it will search for IK groups that 
  are named the same as the arm and with consecutive numbering (left\_arm, left\_arm1,
  left\_arm2,...).
  That allows to reconfigure the IK calculation module and introduce additional IK groups without 
  touching the plugin code. It is expected that at least one IK group is configured, otherwise
  an error message is written to the console.
  
  The collision status of the arm is determined by using the configured collision detection module.
  On each simulation step the collision group that is responsible for detecting direct collisions is 
  handled first. If that one detects a collision, a direct hit is reported and it is not necessary
  to handle the second group at all, because a direct hit always implies a hit with the shield as
  well. Only if no direct hit was detected, the second group is handled. The outcome can be queried
  as the current collision state of the arm. During initialization it is searched for a collision
  group that has the same name as the arm, responsible for direct hit detection and a group that is
  named with the naming pattern [arm\_nameShield], responsible for shield hit detection. If one or
  both of the required groups cannot be detected, an error message on the console will be stated and
  the collision detection functionality will not work as expected. Collision state is evaluated on 
  each simulation time step.
  
  Additional items that have to be identified within the model tree are the 7 joints, the end
  effector tip dummy, the IK target dummy and the force sensor on the last link of the arm. The
  origin of the reference frame can be defined by creating a dummy object within the scene with
  the special name 'ref\_frame\_origin'. If such a scene object is found, all Cartesian poses are
  taken with respect to the reference frame of that object, otherwise the poses are interpreted
  absolute within the world reference frame. All the necessary data that is published by the 
  controller can be accessed via this class.
  
- LWRArmController
  The LWRArmController provides the implementation of the Kukie interface. This interface is created
  by Simon Hangl especially for the KUKA arms in the IIS Lab.
  
\subsection{Identifying simulation components in the scene}

As each scene is a hierarchy of various types of scene objects there had to be found a way how to identify subtrees within this hierarchy that belong to known simulation components and should therefore be handeled by the plugin. One way would be to give each part a clearly defined, unique name and then search for those names within the scene. But than it would be necessary to hardcode the name of each single joint name and this is not a preferable solution for this problem. If somebody accidentally changes a name in the model tree then the solution is broken because the plugin looses connection to the underlying object and cannot control it any more. Here V-Rep's custom developer data functionality comes into play. It is possible to put auxiliary data segments to each single object in the scene.(TODO: show image of data segments) This data gets serialized together with the object and can be read programatically. Each data segment starts with a header number which is used to uniquely identify the data from a specific developer. The second element is an integer value that holds the length of the data segment and then comes the data itself. The format of the data can freely be chosen by the developer. It was decided to use string representations of key/value pairs, seperated by a colon (:). The key uniquely identifies the type of component (arm joint, hand joint, IK target dummy,...). The value segment can be used to provide additional data like for example the name of a joint. The left arm's model base for example is tagged with `2497,1:left\_arm'. This tag data item identifies that element as the model base of a LWRArmComponent with the name `left\_arm'. When actualizing for scene content change, the ComponentContainer traverses the scene hierarchy and looks for objects, that are tagged as known components. On success, it creates the specific instance with the object handle of the underlying scene object. During the initialization, the concrete SimulationComponent instance searches then the model subtree for all the necessary parts (joints, dummies, force sensors...). The implementations for the arm and the hand model provide feedback output on the console window about the success of this initialization process.

\subsection{Creating startup launch file}

Describe startup script, environmental variable (VREP\_PACKAGE\_PATH), launch file, how to use different simulation scenes.

\subsection{Documentation and usage examples}
Detailed documentation and implementation details should go into the Appendix


% Motion planning first part
%!TEX root = thesis.tex

\chapter{Motion planning}
Chapter overview

\section{Introduction}

\citep[p. 1--11]{choset2005} describes motion planning as to be the task of finding a collision free path from one robot \emph{configuration} to another one. The classic path planning problem is the so called \emph{piano mover's problem}, originally mentioned by \cite{schwartz1983}. It is assumed to have a piano, which states a three dimensional rigid body and a set of known obstacles. The problem is to find a continuous motion that moves the piano from it's current position to a given target position without touching any of the obstacles. Thereby the piano can freely be moved and rotated in Cartesian space. 

A generalized version of the \emph{piano mover's problem} is to find paths for a robot, composed from a set of rigid bodies, linked by joints while enforcing \emph{constraints} during that motions. A \emph{constraint} could be to avoid obstacles or to keep the robot's end effector in an upright position. Therefore it is important to have a representation of a robot's state that allows to determine the location of all robot parts. This representation is called the \emph{configuration} of a robot and the \emph{configuration space} is the set of all possible configurations, the robot is able to acquire. The dimension of the configuration space is the amount of \emph{degrees of freedom} (DOF), which is the number or independent variables that are necessary to describe a configuration. An imaginary free flying piano has six degrees of freedom as it's configuration consists of the position and orientation $(x,y,z,roll,pitch,yaw)$ in Cartesian space. A robot arm with 7 joints has 7 degrees of freedom and it's configuration are the joint positions. The motion planning problem is to find a curve in the configuration space that connects start and goal configuration without violating constraints. This is a very complex problem and there exist various different approaches to find solutions. Examples are among others the \emph{bug algorithms}\citep[chapter 2]{choset2005}, \emph{potential functions}\citep[chapter 4]{choset2005} or \emph{sampling-based methods}\citep[chapter 7]{choset2005}. As the solution within this project only uses sampling based algorithms, a short overview about this class of methods will be given in the following section.

\section{Sampling-based motion planning}

Based on \citep[chapter 2]{omplPrimer}, sampling-based motion planning can be seen as a powerful concept, capable of handling planning problems efficiently, especially for systems with many degrees of freedom. The general idea is to generate a uniform set of random sample points in the configuration space and then connect start and goal state by connecting the samples via collision free paths, with respect to possible motion constraints. Those methods are usually faster than traditional approaches because it is not necessary to reason about the whole configuration space but only about a finite number of sample configurations. The majority of sampling-based approaches are known to be \emph{probabilistic complete}, which means that the probability of finding an existing solution tends to 1 as the number of sample points increases to infinity. But they are not able to decide if a valid solution exists at all. The following definitions are used throughout this section to describe the concepts of sampling-based motion planning.

\begin{itemize}

\item \textbf{State space} \\
The \emph{state space} $\mathcal{S}$ is equal to the configuration space and consists of all possible robot configurations (states).

\item \textbf{Free state space} \\
The \emph{free state space} $\mathcal{S}_{free}$ is a subset of $\mathcal{S}$, containing only collision free states.  

\item \textbf{Path} \\
A \emph{path} is a sequence of states. If each state within the path is contained in $\mathcal{S}_{free}$, it is called a \emph{collision free} path.  

\end{itemize}

The sampling-based motion planners can be categorized into two major types - \emph{probabilistic roadmaps} (PRM) and \emph{tree-based planners}. Common to both methods is that they create uniformly distributed samples within the free state space. As the shape of $\mathcal{S}_{free}$ is not explicitly known, the created sample states are checked for collisions before using them. The following paragraphs give a short overview about both approaches.

\paragraph{Probabilistic roadmaps}

That approach uses the sampled states to create a ``roadmap'' of the free state space. Therefore each sample point is connected to an amount of $k$ nearby sample points via collision free paths. This is done by a local planner that simply interpolates between two points in the desired resolution while watching out for collisions. If no collision is detected, a new edge is introduced into the graph that is formed by the roadmap. After completing the graph, a planning query can be reduced to finding the shortest path within that graph that connects the start state and the goal state. (Images will follow...)

\paragraph{Tree-based planners}

A lot of different sampling-based planning algorithm are using the tree based approach, as there are for example (RRT, EST, SBL or KPIECE). The difference to PRM is that this method uses a tree data structure of the free state space, which means that the resulting graph contains no cycles. The root of the tree is the start state and the tree is then expanded towards the goal state by creating collision free connections between the sample points. If the goal is reached, the solution is found. The different approaches differ in the strategy that is used to expand the tree towards the goal state. (Images will follow)\\

Generally can be said that the tree based approaches are more adequate for \emph{single query planning} because the tree usually does not has cover the whole free state space. A roadmap could be reused for subsequent queries. As the most methods also require the search of a nearest neighbour, the utilized distance metric is also a crucial part within sampling-based motion planning as it is not always easy to identify the optimal method for finding nearby states in systems with large degrees of freedom.  

\section{The MoveIt! motion planning framework}

MoveIt\footnote{http://moveit.ros.org} is an open source framework for motion planning. It is the successor of the previous arm navigation stack and therefore fully integrated into ROS. MoveIt was originally developed by Willowgarage\footnote{http://www.willowgarage.com} but since April 2012 it is maintained by the Open Source Robotics Foundation (OSRF). Figure\ref{fig:moveit_arch} gives an overview about the system architecture. Central part of the framework is the \texttt{move\_group} node which  provides a rich set of ROS topics and services that can be used to solve planning problems and execute calculated motion plans on connected hardware. Configuration of that node happens via the parameter server. It requires a complete kinematic and semantic description of the robot setup and a lot of other configuration parameters which will be described in subsequent sections. The most important parts of MoveIt are implemented as plugins which means that it is highly customizable. By default it uses the Kinematics and Dynamics Library\footnote{http://www.orocos.org/kdl} (KDL) as IK solver and the Open Motion Planning Library (OMPL) as planning plugin but it is possible to replace them with custom solutions if necessary.\\

\begin{figure}[ht]
	\centering
  	\includegraphics[width=0.75\textwidth]{images/moveit_architecture.jpg}
	\caption[Moveit architecture]{MoveIt architecture}
	{\scriptsize Image source: \url{http://moveit.ros.org/documentation/concepts}}
	\label{fig:moveit_arch}
\end{figure}

MoveIt maintains a planning scene which is an internal representation of the world, including the robot and it's environment. The base is a description of the robot and it's various planning groups. 
The kinematic description needs to be available in form of a URDF model. Necessary steps to create that model are described in Section\ref{sec:urdf}. Based on that URDF model, a semantic description (SRDF) is created that contains additional information about the setup, as explained in Section\ref{sec:moveit_assistant}. On top of those static descriptions, MoveIt allows to modify the planning scene on runtime via appropriate ROS topics.\\

To keep track of the current state of the robot MoveIt needs to be informed continuously about actual joint states. This is done via publishing the joint states to a specific topic. If there are additional objects within the robot's workspace they also have to be added to the planning scene. This can be done either by explicitly adding them via the corresponding topic or by integrating sensor information like Kinect camera data. MoveIt can then take those objects into account during motion planning and avoid collisions. But some collisions are intended. For example if an object has to be picked up, the gripper has to get in contact to this object. That means, collision checking for specific objects has to be (temporary) disabled to allow those controlled collisions. Therefore MoveIt maintains an \emph{AllowedCollisionMatrix} to which objects can be added or removed. The connection to the hardware happens via the \emph{FollowJointTrajectory} action interface. Each robot component has to provide this interface if it is intended to be control via MoveIt. This interface consists of a set of ROS topics that provide trajectory execution functionality and also allow monitoring the current execution status. The major contribution within this chapter to the whole project was to provide an implementation of this interface and connect it to the one that is currently available.

\section{Creating the URDF model of the robot setup}
\label{sec:urdf}

The \emph{Unified Robot Description Format} (URDF) is a markup language, designed to describe robots. The description happens in text files, in a special XML format. The most important elements in the XML specification\footnote{http://wiki.ros.org/urdf/XML} are:

\begin{itemize}

\item \textbf{<link>} \\
Describes the all necessary properties of a specific robot link. Each link must have a unique name. The visual, inertial and collision details are configured in the corresponding subtags of the link element. The visual part as well as the collision model can either be composed from primitive shapes
or from mesh files. If mesh files are used it is important that they are not to complex. Especially for the collision model it is recommended to use a simplified model to avoid a performance loss.

\item \textbf{<joint>} \\
Describes the properties of a joint. A joint is a connection between two links, having exactly one parent and one child link. Each joint states a new reference frame for it's child link and it is positioned relative to it's parent frame. A \emph{fixed} joint is a rigid connection between parent and child link. There are different types of joints available but for the current project only \emph{revolute} joints\footnote{Rotational joint with one degree of freedom} are of interest. Details like limits, axis orientation and dynamic properties can be configured in the corresponding subtags of the joint element.

\end{itemize}
\begin{figure}[ht]
	\centering
  	\includegraphics[width=0.75\textwidth]{images/urdf_chain.jpg}
	\caption[URDF graph]{URDF graph}
	{\scriptsize Image source: http://wiki.ros.org/urdf/Tutorials/Create your own urdf file}
	\label{fig:urdf_graph}
\end{figure}
Those elements are used to form the URDF graph that exactly describes the kinematic chain of the robot components and their placement relative to each other as visualized in Figure\ref{fig:urdf_graph}. This description can get very large as a lot of different components are involved. So it is possible to organize it into a set of text files, each one describing one part of the whole. For example one file describes the arm itself. Other files can then use that description and insert multiple instances of that arm. The \texttt{xacro} ROS package provides the necessary functionality to combine all those text files into one XML string. \emph{XACRO} stands for \emph{XML Macro} and is designed to parse xacro files and combine them into one single XML document, containing the resulting URDF description.\\

The URDF description of the IIS robot setup is spread across multiple packages, located in the \texttt{iis\_hardware} stack. Arm and gripper descriptions are located in seperate packages (\texttt{lwr\_description} and \texttt{schunk\_description}), the \texttt{iis\_robot} package brings all the components together. The modelling process started with the search for pre-existing URDF descriptions of the required robot components. A suitable description of the Schunk SDH gripper was taken from the \texttt{schunk\_description} ROS package. A model of the KUKA LWR arm was found in the Github repository\footnote{https://github.com/RCPRG-ros-pkg/lwr\_robot/tree/hydro-devel/lwr\_defs} of the \emph{Robot Control and Pattern Recognition Group}\footnote{University of Warsaw (http://robotyka.ia.pw.edu.pl/twiki/bin/view/Main)}. The other parts of the model, namely the robot torso and the table had to be created. The descriptions are located in a seperate files within the \texttt{uibk\_robot} package (\texttt{torso.xacro} and \texttt{table.xacro}).

The mesh files, used in the description of the robot torso have been exported from the V-Rep simulation scene. For the visual part, the mesh was taken as it is. For the collision model, the width of the original mesh was increased to provide some safety padding. Moreover a cylinder with a diameter of 40cm was added in the head area to ensure that the planner avoids accidental hits in this sensitive region. Figure\ref{fig:torso_col} shows the visual and the collidable part of the torso model.
\begin{figure}
	\centering
  	\includegraphics[width=1.0\textwidth]{images/torso.jpg}
	\caption{Left image shows visual part, right image the collidable part of the torso}
	\label{fig:torso_col}
\end{figure}
The table was modelled from primitive shapes. It was taken care that the size of the table can easily be adjusted, as can bee seen in Listing\ref{lst:table}. The collision model of the table is also slightly larger than the visual part to provide some safety margins.
\lstset{language=XML,style=customxml}
\begin{minipage}{\linewidth}
\begin{lstlisting}[caption={XML snippet, inserting the table model into the URDF}, label=lst:table]
<!-- draw the table relative to the origin -->
<xacro:model_table name="table" 
	     parent="world"
	     length="2.22"
	     width="0.8">
	<!-- Place the table relative to the world reference frame -->
	<origin xyz="-0.029 -0.3 0" />
    
</xacro:model_table>
\end{lstlisting}
\end{minipage}

When attaching the two grippers it showed that the offset between gripper wrist and last arm link was not correct. To correct that issue the file \texttt{sdh\_with\_connector.xacro} was created which simply places an additional ring between the last arm link and the gripper.\\

The file \texttt{iis\_robot\_table.xacro} draws all the pieces together. It describes the whole setup, consisting of the torso, two arms, two grippers and the table. The root element of the model hierarchy is a link called \texttt{world\_link}. Torso, table and both arms are positioned relative to that root link. Changing the world reference frame could easily be achieved by shifting the root link to a new position. Figure\ref{fig:robot_table} shows a visualization of the URDF description.
\begin{figure}
	\centering
  	\includegraphics[width=0.75\textwidth]{images/iis_robot_table.png}
	\caption{URDF description in RViz}
	\label{fig:robot_table}
\end{figure}

\section{Configuring the planning tools}
\label{sec:moveit_assistant}

The MoveIt configuration package is created, using the \emph{MoveIt Setup Assistant}. Precondition is an existing URDF description of the robot setup which was created in the previous step.
The setup process comprised of the following steps:

\begin{itemize}

\item \textbf{Computation of the self collision matrix}

The self collision matrix consists of pairs of robot links that can safely be excluded from collision checking. Neighbouring links for example are in permanent collision. Collisions between other links can never happen because they are simply too far apart. The Setup Assistant calculates a large number of different robot configurations and tracks for link pairs that are mostly always in collision and pairs that are never in collision. The self collision matrix can be adjusted manually if necessary. Excluding a large number of link pairs raises performance during motion planning because collision checking is an expensive process.

\item \textbf{Defining the planning groups}

Each planning request in MoveIt is done against one of the defined \emph{planning groups}. A planning group is a group of links and joints within the model that can be seen as one logical component. Planning groups are defined for both arms and grippers. The configuration for the arms also requires the definition of the utilized IK solvers but currently only the default KDL solver is available. An additional planning group, called \texttt{both\_arms} allows planning requests for both arms simultaneously.

\item \textbf{Defining the end effectors}

End effectors are the left and the right gripper. Each end effector has a name and consists of one of the predefined planning groups, a parent group and the parent link which is the last link in the kinematic chain of the parent group.

\item \textbf{Generate the configuration files}

After completing all the configuration steps the configuration package can be generated. Therefore a package name has to be specified which was set to \path{uibk_robot_moveit_config}. This package is a prototype, containing a large number of predefined configuration files. Some of those files have to be extended manually, adding additional information about available robot controllers and sensors. 

\end{itemize}

Completing the last step results in a ROS package, containing necessary configuration and launch files for the given robot setup. The semantic robot description can be found in the \path{iis_robot.srdf} file. It contains previously defined parameters like the self-collision matrix, planning groups and end effectors. The configuration package can be tested, running the following command on the command line:
\begin{quote}
\begin{verbatim}
roslaunch uibk_robot_moveit_config demo.launch
\end{verbatim}
\end{quote}
This commands launches a \path{move_group} node in demo mode, using the previously created configuration files and starts an instance of RViz with the motion planning plugin. There it is possible to switch between the planning groups, set start and target configurations, do planning requests and visualize the outcome. The setup can be modified by launching the Setup Assistant again, using the \path{setup_assistant.launch} file from this package.

\section{Connecting MoveIt to the existing robot control interface}

Now MoveIt is configured and ready to handle planning requests for the robot setup. But execution of the resulting trajectories is still impossible because of the missing connection between MoveIt and the involved robot hardware. Each component that is intended to be controlled by MoveIt needs to provide the \emph{FollowJointTrajectory} action interface, as defined in the \texttt{control\_msgs} package. This is a special kind of ROS interface that allows to send trajectories to robot components and monitor the execution status. As the existing control interface of the IIS robot components does not fulfil these requirements, an additional node is necessary that is capable of executing trajectories, using the existing infrastructure.\\

The planning outcome of MoveIt is a time parametrized trajectory, expressed by the \emph{JointTrajectory} message type. This message consists of a set of waypoints. A waypoint is a joint configuration, described by the tuple $(p,v,a,t)$ where $p \in \mathbb{R}^n$ are the positions, $v \in \mathbb{R}^n$ the velocities and $a \in \mathbb{R}^n$ the accelerations at time $t$ and $n$ is the number of involved joints. Those waypoints mark the important points along the path, the manipulator has to move. The controller needs to be able to translate this trajectory into a sequence of suitable motor commands. Just sending the joint positions within the waypoints to the robot would not suffice as a trajectory is also constrained in terms of \emph{velocities} and \emph{accelerations} over \emph{time}. Therefore the controller needs to be able to interpolate between the subsequent waypoints and calculate intermediate joint positions based on the loop rate of the control cycle. This interpolation is a difficult task and the implementation would be far beyond the scope of this project. Luckily the required functionality is already available within the \texttt{ros\_control} stack\footnote{http://wiki.ros.org/ros\_control}. Those packages are created to integrate and control robot hardware in a generalized way, facilitating the usage of existing controllers on various different robots. 

\subsection{ROS control stack overview}

\begin{figure}
	\centering
  	\includegraphics[width=1.0\textwidth]{images/ros_control.png}
	\caption{ROS control architecture}
	{\scriptsize Image source: http://wiki.ros.org/ros\_control}
	\label{fig:ros_control}
\end{figure}
Figure\ref{fig:ros_control} shows an architectural overview of the ROS control stack. The required infrastructure for using generic controllers is mainly provided by two components:

\begin{itemize}

\item \textbf{RobotHW} \\

This is the abstract base class for robot hardware abstraction. It's main purpose is to send commands to the hardware and retrieve state data. The \emph{RobotHW} also maintains a set of \emph{hardware ressource interfaces}. Each generic controller needs a special kind of interface for controlling joints or read their current state. Concrete implementations have to register the required interfaces. For retrieving state data, a \emph{JointStateInterface} is required. Controllers using that interface can query it for a specific joint and ask for the current state. Sending commands to the joints is done via subtypes of the of the abstract \emph{JointCommandInterface}. The \emph{JointPositionInterface} for example allows to send target positions to registered joints. Other implementations are \emph{JointEffortInterface} and \emph{JointVelocityInterface}. Which interfaces are provided depends on the way how the actual hardware is controlled.

\item \textbf{ControllerManager} \\

The \emph{ControllerManager} is responsible to maintain a set of controllers. It provides a ROS interface that allows to load, start and unload controllers and for switching between them. During the control cycle, the \emph{RobotHW} is forced to read current state from the hardware. Then the \emph{ControllerManager} is triggered to update the active controllers based on the current time step. The commanded values are then sent back to the concrete hardware. This control cycle is intended to run in real time when interacting directly with the hardware. Controllers are usually configured on the parameter server and loaded on demand, using the ROS interface of the \emph{ControllerManager}. Custom controllers can be created by inheriting from the abstract  \emph{ControllerBase} class.  

\end{itemize}

\subsection{Designing the hardware adapter}

The \emph{hardware adapter} is an independent ROS node that acts as connection between MoveIt and the simulated or real hardware. It provides the necessary infrastructure for using generic controllers from the \texttt{ros\_control} packages. Figure\ref{fig:hardware_adapter} gives an overview about the hardware adapter architecture. The \emph{UibkRobotHW} is a subclass of \emph{RobotHW} and represents the connection to the robot. It maintains the complete state of all joints in the connected components. Joints are represented by the \emph{Joint} datatype, consisting of a unique joint name, state parameters and a commanded target position. For controllers, the \emph{UibkRobotHW} class provides a \emph{JointStatesInterface} and a \emph{JointPositionInterface}. Active controllers use those interfaces to access the current state and for commanding target positions. 

The \emph{UibkRobotHW} utilizes a set of \emph{JointStateAdapters}, each one representing a connection to one specific robot component. During the control cycle the \emph{JointStateAdapters} read current joint states from the appropriate topics and send commanded values back to the hardware.
\begin{figure}[h]
	\centering
  	\includegraphics[width=1.0\textwidth]{images/hardware_adapter.jpg}
	\caption{Hardware adapter architecture}
	\label{fig:hardware_adapter}
\end{figure}
The \emph{JointStateAdapters} are created during initialization, based on the configuration settings. After creation, each \emph{JointStateAdapter} waits a certain amount of time for an initial joint states message. If it does not receive such a message before timeout, it will automatically shut down for safety reasons and report an error. This is very important as the \emph{JointStateAdapter} imediately begins to send joint positions after initialization. The position to be sent is initially the currently known position, as long as no other values have been commanded by the controllers. Therefore the initial position always has to be known, otherwise dangerous and rapid robot motions could occur on startup. The adapter configuration happens via the parameter server. Available configuration parameters are:

\begin{itemize}

\item \textbf{\texttt{adapter\_list}} \

Contains a list of all \emph{JointStatesAdapters} that have to be created. For each adapter name mentioned in this list a detailed configuration is required. The subsequent parameters have to be configured for each single adapter.

\item \textbf{\texttt{joint\_state\_topic}} \

The name of the topic where a specific adapter listens for joint states. The expected message type  \texttt{sensor\_msgs/JointStates}. This parameter is mandatory.

\item \textbf{\texttt{readonly}} \

This parameter is optional. If true then the adapter will only listen for joint states but not publish commanded values.

\item \textbf{\texttt{joint\_command\_topic}} \

The name of the topic the adapter should use to publish the commanded values to. The expected message type is \texttt{std\_msgs/Float64MultiArray}. This parameter is mandatory if the adapter is not configured to be read only.

\item \textbf{\texttt{joints}} \

A list of joint names that should be controlled by the adapter. The order in this list also determines the order of the values in the published message.

\item \textbf{\texttt{joint\_name\_prefix}} \

This parameter can be used to be able to uniquely identify joints. For example both arms are using the same joint names (\texttt{arm\_0\_joint}, \texttt{arm\_1\_joint},\ldots). But in a URDF model,  joint names have to be globally unique. Therefore the prefix can be used to prepend the original name to achieve uniqueness. A value of \texttt{right\_} for example will lead to the joint names \texttt{right\_arm\_0\_joint}, \texttt{right\_arm\_1\_joint},\ldots).

\end{itemize}

An example configuration can be found in Listing\ref{lst:adapter_config}. As the topic names, used by the \emph{JointStateAdapters} can be configured freely, the hardware adapter is able to interact with the simulator and the real robot as well, as both of them provide exactly the same ROS interface.

\begin{minipage}{\linewidth}
\lstinputlisting[caption={Adapter configuration for left arm and gripper}, label=lst:adapter_config]{code/adapter_config.yaml}
\end{minipage} \\

During hardware adapter startup an instance of the \emph{UibkRobotHW} and \emph{ControllerManager} is created from configuration. As the control loop must not be interrupted by ROS callback functions, it is launched in a separate thread after completing initialization, using a loop rate of 100hz. On each iteration the exact time since the last step is calculated. The \emph{ControllerManager} then updates all registered controllers based on current state and time since last iteration. After handling the controllers, the \emph{JointStatesAdapters} are forced to send the commanded values to the hardware. The ROS callback functions are handled in the original thread.\\

After starting the hardware adapter, the required controllers have to be loaded and started. This is usually done, by using the corresponding ROS services, provided by a running \emph{ControllerManager} instance, but the \texttt{controller\_manager} package contains a tool named \texttt{spawner} that simplifies the starting process. The configuration of each controller has to reside on the parameter server. Utilized controllers are the \emph{JointTrajectoryController} and the \emph{JointStateController}. The \emph{JointTrajectoryController} provides the required \emph{FollowJointTrajectory} action interface for a group of joints. For each robot component a separate \emph{JointTrajectoryController} is used. The \emph{JointStateController} publishes the collected states of all robot joints at once to the \texttt{joint\_states} topic, which is used by MoveIt to monitor the robot configuration.

\subsection{Launching the hardware adapter}
\label{sec:launch_moveit}

The file \texttt{hardware\_adapter.launch}, located in the \texttt{uibk\_moveit\_adapter} package was created to handle the necessary configuration parameter upload and launch the hardware adapter node for the simulator and the real robot as well. As can be seen in Listing\ref{lst:adapter_config}, the configured topic names for the \emph{JointStateAdapters} are defined, using \emph{relative} graph resource names (i.e. without trailing slash). The required instance is than accessed by simply shifting the node into \texttt{simulation} or \texttt{real} namespace. This is realized, specifying the \texttt{config\_name} parameter of the launch file. The command line statement
\begin{verbatim}
roslaunch uibk_moveit_adapter hardware_adapter.launch config_name:=simulation
\end{verbatim}
launches the hardware adapter in \texttt{simulation} namespace.\\

After launching the hardware adapter node, the required controllers are loaded and started. The node provides feedback information about the state and errors during startup process on the console output. It is crucial that the joint state topics of simulator or real robot are available before launching the hardware adapter node, otherwise it will not be able to work because of missing initial joint states. After successful startup, the additional controller topics can be found in the corresponding namespace.

\section{Adjusting the MoveIt configuration}

The prerequisites are now made for connecting MoveIt to the (simulated or real) robot. The last remaining step is to adjust the MoveIt configuration for establishing a connection to the \emph{FollowJointTrajectory} controllers. Therefore the configuration file \path{controllers.yaml} was created in the \path{uibk_robot_moveit_config} package. It contains a list, describing the available controllers. Each controller description contains the name of the controller, it's action namespace, the controller type and a list, containing the names of the controlled joints.\\

For starting MoveIt, the launch file \path{moveit_planning_execution.launch} was created. This file is intended to launch a \texttt{move\_group} node either for simulated or real robot, together with the corresponding hardware adapter. The namespace can be selected using a boolean parameter named \texttt{simulation}. The statement
{\small 
\begin{verbatim}
roslaunch uibk_robot_moveit_config moveit_planning_execution.launch simulation:=false
\end{verbatim}}
launches a MoveIt configuration, connecting the \texttt{move\_group} instance to the real robot. The launch file also starts RViz configured with the motion planning plugin. This is used for visualizing the planned trajectories and can also be used to test the connection to the robot by making planning requests and execute the resulting trajectories.
% Motion planning practical part
%!TEX root = thesis.tex

\chapter{Pick and place}
\label{chap:pick_place}

The implementation of a benchmark pick and place task, executable on simulator and real robot as well, states one of the objective targets of this project. The implementation heavily uses major parts of MoveIt's planning functionality. The overview at the beginning of this chapter gives some general information about pick and place tasks, explaining the process step by step and describing the single stages that have to be performed. The second part focuses on the pick and place functionality of MoveIt and discusses the involved action servers, topics and messages. The third section explains the implementation of the benchmark pick and place task in greater detail and demonstrates how that functionality of MoveIt was used. The last section describes some observations that have been made during the implementation process.

\section{Overview}

A pick and place task is the process of grasping an object, lifting it and dropping it somewhere else. Humans do that permanently, without even think about it. But when teaching a robot to perform a pick and place action, it shows how difficult and complex this task is and how much planning is involved to achieve the desired result. The planner requires exact knowledge about the robot and it's environment, including the objects to grasp and possible obstacles. Accidental collisions have to be avoided but other collisions are mandatory when the robot has to get in contact with the world. The gripper definitely collides with the object to pick, but only during grasping and holding. Therefore there has to be a mechanism to explicitly tell the planner that specific collisions are allowed during particular stages of the operation. Moreover, after grasping an object it has to be considered as an additional part of the robot during subsequent planning requests. That means that it must be attached to the manipulator temporarily and removed, after releasing the object. As long as the object is part of the robot, it possibly increases the size of the end effector.\\

Stationary objects usually stand or lie on a surface, called the \emph{support surface}. During the interaction with an object, possible collisions with the support surface have to be taken into account. It is also possible that the whole process underlies additional constraints, so called \emph{path constraints}. This type of constraint has to be enforced along the whole path, the grasped object takes during the operation. For example when carrying a glass, filled with liquid it has to stay in an upright position, otherwise the liquid is lost. That means, the glass has to be held in a specific orientation during the whole task. This can be described as \emph{orientation constraint} which is a special type of path constraint. The planning solution has to provide mechanisms to define end enforce such types of \emph{path constraints}. \\

Pick and place tasks can be split into two independent phases, each one composed from a number of trajectory stages:

\begin{itemize}

\item \textbf{Pickup phase} \\

This phase starts at an arbitrary robot configuration. In the first stage, the manipulator has to be brought into a position closed to the object to grasp, but in a distance that allows the gripper to open safely without touching the object. This position is called the \emph{pre-grasp pose}. The next stage is to set the gripper into \emph{pre-grasp posture}. That means, bring it's fingers into an open configuration that allows to completely enclose the object (or at least that part that is used to clutch it) after approaching towards the final \emph{pose}. How this configuration looks like depends on the shape of the object to grasp and the structure of the gripper. Further on, the gripper has to be moved towards the \emph{grasp pose} using the correct approach direction. This is the place where the robot gets in contact with the object. The gripper moves it's fingers into the \emph{grasp posture} - a configuration that encloses the object and applies as much force as necessary to be able to take and hold it. The resulting collisions between the gripper links and the object have to be ignored. At that point the object has to be attached to the gripper and further on treated like an additional link of the robot although still being in collision with the support surface. The pickup phase completes after lifting the object along the retreat direction. The object is now part of the robot and no collisions should be tolerated any more. Figure \ref{fig:pickup} shows the stages of the pickup phase.

\begin{figure}[ht]
	\centering
  	\includegraphics[width=0.75\textwidth]{images/pickup.jpg}
	\caption{Stages of the pickup phase in the simulator}
	\label{fig:pickup}
\end{figure}

\item \textbf{Placement phase} \\

The placement phase starts after a successful pickup. The grasped object is enclosed by the gripper and considered to be part of the robot. Now the manipulator moves towards the \emph{pre-place location} - the place where the final approach towards the goal starts. The easy way is to just drop the object at that point. In that case, the placement phase completes after bringing the gripper fingers into the \emph{post-place posture} (opening it) and detaching the object from the robot.

If the object should to be placed carefully, the gripper has to approach from the \emph{pre-place location} towards the final \emph{place location} along the specified approach direction. Here has to be considered that the object will get into contact with the support surface again when it's final position is reached. At that stage the gripper can open and the object has to be detached from the robot. From that point, the object needs to be treated as obstacle again which means the planner is forced to avoid collisions with it. The placement phase completes after the manipulator has moved away from the object along the specified retreat direction. The stages of the placement phase can be seen in Figure \ref{fig:placement}.

\begin{figure}[ht]
	\centering
  	\includegraphics[width=0.75\textwidth]{images/placement.jpg}
	\caption{Stages of the placement phase in the simulator}
	\label{fig:placement}
\end{figure}

\end{itemize}

Both phases of the task are only considered to be complete if each single stage has successfully been executed. The necessary planning parameters like pre-grasp and grasp pose, gripper postures and approach and retreat directions are usually provided by a \emph{grasp planner}. This is an additional node within the planning pipeline that identifies objects in the environment of the robot, usually based on 3D sensor data and calculates the necessary grasp parameters. Explaining the functionality of grasp planners is far beyond the scope of this project though the section about implementing the reference task discusses the used parameters in greater detail and shows what would be usually delivered by the grasp planner.


\section{Pick and place tasks in MoveIt}

During the various stages of the pickup and placement phases a lot of motion planning is required. Each single stage requires to plan a trajectory that is free of accidental collisions while respecting the limits of the robot and enforcing possible additional constraints. Each phase can only be considered executable if a valid motion plan for each single stage exists. Planning all the subsequent stages one after the other would be a cumbersome task. Therefore MoveIt provides a set of messages and action servers that greatly simplify those planning tasks.\\

The \emph{PickupAction} server handles the planning and optionally execution of all required stages during the pickup phase at once. Pickup requests are done, using a \emph{PickupAction} client to send \emph{PickupGoal} messages to the server. The request message is composed of all the parameters that are necessary to completely describe the planning problem. Picking up an object can always be done in several ways. Depending on the shape of the object there hardly always exist a lot of possible poses where the gripper can safely approach and grasp. Therefore MoveIt allows to provide a set of possible grasp definitions for a pickup request. There is also a quality parameter that can be used to tell the planner, how `good' a specific grasp is compared to other ones. MoveIt can then favour grasps with higher quality if several valid solutions are found by the planner. Providing a number of different grasps increases the probability for the request to be successful. The grasp definition is composed of the pre- and post grasp postures for the gripper, the final grasp pose and the approach and retreat vectors. As MoveIt needs to know which object should be picked within the planning scene, it is also necessary to include the ID of the object to grasp.\\

The \emph{PlaceAction} server is responsible for planning the whole placement phase. Therefore same concept is used as above - a set of possible place locations can be provided to increase the probability of a successful planning attempt. The parameters that are used to describe a place location include the post-place posture of the gripper along with it's target pose, and vectors, describing the pre-place approach and the post-place retreat.\\

Pickup and placement requests as well require a number of additional parameters that are explained in greater detail within the section about benchmark task implementation. 

Objects contained in the task environment have to be brought to MoveIt's attention. This can be done either by manually adding them to the planning scene, using the corresponding topics or by configuring MoveIt to be aware of sensor data. As the integration of depth information from the Kinect camera did not work stable during the evaluation phase, that possibility is not covered within this thesis and all involved objects were added manually.

Pickup- and place action servers provide the ability to choose whether to execute motion plans immediately or to do just the planning and return the outcome. In that case the execution has to be handled manually later on. 
The first method is much more comfortable as MoveIt executes the trajectories and also handles additional requirements like attaching and detaching the grasped object in time. The drawback is that also possibly weird trajectories get executed immediately as they might be valid solutions for the given planning problem though they are obviously unsuitable. So an additional safety mechanism needs to be introduced that allows to interrupt the execution when facing any problems.
The second method allows to visualize the resulting trajectories and then decide whether to execute them or not. But then each single trajectory stage has to be executed manually which also includes attaching or detaching objects to the manipulator. The advantage of this method is the clean separation between planning and execution, allowing maximum control over the execution flow. Therefore this method was favoured during benchmark task implementation.

\section{Implementation of the benchmark task}

This section describes the implementation of the benchmark pick and place task in greater detail. The source code can be found within the `uibk\_moveit\_tests' package. The workspace is the table in front of the robot covered with a 9cm thick foam mat. This mat will be declared as the support surface later on. The object to grasp is a cylinder with 4cm radius and a height of 25cm. This corresponds to the size of a usual SIGG bottle which can be used to run the benchmark task in the real world. The cylinder is located on a fixed, known position. There is also a box shaped object located within the workspace that acts as an additional obstacle. The goal of the task is to pick the cylinder up, using the right arm of the robot and place it at the goal location without colliding with the obstacle or other parts within the robot's environment. The implementation makes use of MoveIt's pick and place functionality. The necessary steps are explained in the following subsections.

\subsection{Creating the environment}

As MoveIt is currently not configured to use sensor data, it has to be notified about the task environment. The table and the surface mat are already part of the URDF description but the cylinder and the obstacle have to be added to the task environment. This is done utilizing the \texttt{planning\_interface::PlanningSceneInterface} class that provides functionality to manipulate the current planning scene. The same effects can be achieved by manually publishing to the `/collision\_object' topic but using the convenience class saves a lot of boilerplate code. The \emph{CollisionObject} type is used to describe those objects. Both of them are primitive shapes. Necessary parameters are the shape type, dimensions and the target pose. Additionally each \emph{CollisionObject} needs a unique ID which is used to identify the shape within the planning scene. After adding the \emph{CollisionObjects} to the collision world, they are visualized in RViz. The image in Figure\ref{fig:task_env} shows the task environment after adding the \emph{CollisionObjects}.

\begin{figure}[ht]
	\centering
  	\includegraphics[width=0.75\textwidth]{images/task_env.png}
	\caption{Planning scene after inserting the collision objects}
	\label{fig:task_env}
\end{figure}

\subsection{Generating possible grasps}

Before calling the pickup action server it is necessary to generate a set of possible grasps for the object to pick. This information is usually provided by a grasp planner. The sample task encapsulates the grasp planner functionality within the \emph{generateGrasps} method. This method takes the current pose of the cylinder as input parameter and calculates 10 possible grasp poses along a semi circle around the cylinder location. Pre-grasp and grasp postures are joint trajectories for the gripper, containing just one trajectory point stating the target configuration for the gripper in the opened respectively the closed state. The pre-grasp approach and the post-grasp retreat are defined as \emph{GripperTranslations}. This is a special message type that describes the direct gripper movement from one position towards a target. The direction is defined as a three dimensional vector. The length of the translation can be set in a flexible way by specifying a desired distance and a minimum distance. Experiments showed that the success rate is much better if the grasp parameters allow some flexibility to the planners. The approach vector depends on the grasp pose and points along the z-axis of the end effector frame towards the object. The desired distance between pre-grasp and grasp pose is set to 20cm while the minimum distance is 10cm. The gripper retreat vector points up, along the z-axis of the world. Desired and maximum distances are also set to 20cm respectively 10cm. As no one of the generated grasps should be favoured among the others, the grasp\_quality parameter of each grasp is set to 1. This means that they are equal in quality. The last necessary parameter is a unique identifier for each grasp. As usually a lot of different grasps are provided for each pickup request, this ID can be used to identify, which one was finally used to solve the planning problem.

\begin{figure}[ht]
	\centering
  	\includegraphics[width=0.75\textwidth]{images/grasp_stages.jpg}
	\caption{Grasp poses, gripper approach and retreat}
	\label{fig:grasp_stages}
\end{figure}

\subsection{Planning and executing the pickup}

After generating the set of grasps, the planning request can be sent to the pickup action server. As there is a lot of boilerplate code necessary, a reusable helper class was created that can be used to simplify all kinds of planning requests. The utility class is called \emph{PlanningHelper} and can be found within the 'uibk\_planning\_node' package. A call to the pickup action server is done by using the `plan\_pick' method of the helper class. This method takes a set of possible grasps and the ID of the object to pick as parameters. The outcome is a pointer to an instance of \emph{PlanningResult}, a structure that contains all the necessary information about a planning attempt. Success or failure is indicated by the `status' parameter. On success, the parameter `trajectory\_stages' holds a vector, containing the resulting trajectory stages. The actual call to the pickup action server is done by defining the  \emph{PickupGoal}, using the predefined grasps. Additional parameters are the ID of the object to pick, the name of the chosen planning group and the name of the link within the robot model that acts as the support surface. Optional parameters are among others the ID of the planner to use and the maximum allowed planning time. The parameter `plan\_only' within the planning options is set to true to avoid the immediate execution of the planned trajectories. This allows a visual verification of the planning outcome before execution. The resulting robot path is shown in RViz. If the solution is satisfying it can be executed, passing the PlanningResult to the corresponding method of the PlanningHelper class. This method uses the `/execute\_kinematic\_path' service provided by the `move\_group' node. The service sends a given trajectory to the responsible controller and provides feedback information about the execution status. The `PlanningHelper' also takes care to attach the picked object to the gripper after the grasp stage. The pickup phase completes after successful execution of all trajectory stages.


\subsection{Planning and executing the placement}

The placement phase is planned and executed in a somehow similar manor. The cylinder has to be placed on a specific location within the workspace in an upright position - the rotation around the z-axis doesn't matter. Therefore a set of possible place poses is generated in 20 different orientations around the z-axis. This again gives some freedom to the planner as it can choose which one to use (TODO: provide image that shows the principle). The final approach towards the goal location is specified as \emph{GripperTranslation}. The direction vector points down along the z-axis of the world. Desired and minimum distances are set to 20cm respectively 10cm, allowing some flexibility during that stage. As post-place posture for the gripper, the same configuration is used as for the pre-grasp posture. The last required parameter for a place location is the \emph{GripperTranslation} that describes the retreat after releasing the object at the goal location. The direction depends on the chosen orientation and points towards the negative z-axis of the gripper reference frame. 

\begin{figure}[ht]
	\centering
  	\includegraphics[width=0.75\textwidth]{images/place_stages.jpg}
	\caption{Place locations, pre-place approach and post-place retreat}
	\label{fig:place_stages}
\end{figure}

The call to the place action server is again handled by the formerly mentioned planning helper class. The `plan\_place' method takes a vector of the predefined place locations and the ID of the object to place as input parameters. The actual call to the place action server is done, defining a \emph{PlaceGoal} message. The most important parameters are the ID of the object to place, the name of the planning group and the set of possible place locations. Optionally can be specified which planner to choose and also the maximum allowed planning time. The method returns the result of the planning request. On success, the resulting trajectories are again visualized in RViz and can then be executed the same way as during pickup phase. After the execution, the task is considered to be complete and the picked object is detached from the gripper and again part of the collision world.


\section{Executing the benchmark task}

The benchmark task is designed to run on the simulator and the real robot as well. The sample program only depends on a running `move\_group' instance. The easiest way to run the sample is to start MoveIt in demo mode (demo.launch) and then execute the sample code, using.

\begin{verbatim}
rosrun uibk_moveit_tests sample_pick_place 
\end{verbatim}

This demonstrates the functionality but only executes the trajectories on the fake controllers provided by MoveIt. If the sample should be executed on the simulator or the real robot, the corresponding namespace has to be specified before running the code. This can be done by setting the `ROS\_NAMESPACE' environment variable to either `simulation' or `real', within the terminal. For example

\begin{verbatim}
export ROS_NAMESPACE=simulation
rosrun uibk_moveit_tests sample_pick_place 
\end{verbatim}

runs the benchmark task on the simulator. Of course it is necessary to start the simulator and launch the corresponding MoveIt instance before doing that. After creating the environment and adding the objects to the planning scene the pickup phase gets planned. The outcome can be seen in RViz and the program will ask if the resulting trajectory is ok and should be executed. A negative answer will force the program to replan the pickup and ask again, otherwise the trajectory gets executed. After successful execution, the place phase gets planned. The resulting plan also is the visualized as well and execution needs confirmation again. The program exits after successful completing the placement phase.

\section{Observations}

This section gives an overview about the most important observations that have been made during the implementation of the sample task:

\begin{itemize}

\item

It is very important to provide some degree of freedom to the planner at various points. Major points are the amount of different grasps or place locations and the allowed range within the defined gripper translations. Very strictly defined planning requests are very likely to fail whereas requests with a higher degree of flexibility drastically raise the overall success rate.

\item

Working with path constraints drastically drops the success rate because the high complexity of the planning problem. Enforcing path constraints requires much more allowed planning time and a very fast IK solver because of the large number of necessary IK requests. Maybe a faster IK solution than the one available could help to solve the problem but that is not guaranteed. Therefore there are no path constraints used within the sample task.

\item

Planning requests often fail due some MoveIt-internal issues and not because planning is not possible at all. Therefore it is necessary to repeat failed requests because it is very likely that planning succeeds on subsequent attempts. In the sample task implementation, the planning requests are done within a loop. If a request fails, it gets repeated. A successful request breaks the loop and continues the execution flow.

\item

Resulting trajectories should always be visualized in RViz before confirming the execution. The calculated motion plans might be valid but sometimes they are a bit confused and therefore unsuitable. Moreover, the planner can only take into account what it knows about the robot's environment. Especially in the robot lab there is equipment mounted in the area above the robot, that is not taken into account during planning. Therefore the visual validation is necessary to avoid damages on the robot and it's environment.

\end{itemize}

%\lstset{language=C++}
%\begin{lstlisting}

%This is code

%int a = 0;
%printf("Dies ist ein Test");

%\end{lstlisting}

%\lstinputlisting[language=C++,
%				 directivestyle={\color{black}},
%				 emph={int,char,double,float,unsigned},
%				 emphstyle={\color{blue}}
%				]{code/move_goal.cpp}





% Conclusion, possible further work, open questions
%!TEX root = thesis.tex

\chapter{Conclusion}

During the scope of this bachelor project, a custom simulation solution for the IIS lab robot setup was designed, utilizing the V-Rep simulation platform. It was shown, how the simulation scene was assembled and how dynamic models of required robot components are created. The modelling process was shown on the example of the Schunk SDH2 gripper component. The resulting reference scene reflects the current setup in the IIS lab, but can easily be adjusted to reflect other configurations. The ROS interface is not tied to a specific scene but to known components. Therefore it is easy to create additional scenes with different setups.

The corresponding ROS control interfaces for the robot components were implemented as a V-Rep simulator plugin. The plugin is loaded on V-Rep startup and identifies known components within simulation scenes. As soon as such a component is identified, it can be controlled by the proper ROS control interface. The plugin design allows to introduce additional components and extend it's functionality. Allows to visualize accidental collisions.

The second objective of that project was to integrate the motion planning framework MoveIt into the current setup. Therefore a URDF representation of the robot setup was created that exactly describes the involved robot components and their placement within the planning environment. Based on that URDF description, the planning framework was set up and configured. The integration process also required the implementation of an additional ROS node that allows the proper execution of time-parametrized trajectories. That node was designed to communicate with the existing ROS control interface and can be configured to interact with the simulator and the real robot as well. 

Finally, the proper functionality of the planning tools and the simulator was shown on a reference \emph{pick and place} task that can be executed on the simulator and the real robot as well.


\bibliography{refs}% Remember to compile the bibliography using bibtex.

\appendix% Start your appendices with this command.
% Implementation details, documentations...
%!TEX root = thesis.tex

\chapter{Simulator Plugin documentation}

%!TEX root = thesis.tex

\chapter{URDF descriptions}
\label{app:urdf}

\lstset{style=customxml}

\section{Complete model with table}
\lstinputlisting[caption={iis\_robot\_table.xacro}, label=lst:urdf_robot_table]{code/iis_robot_table.xacro}

\section{Adjusted gripper model}
\lstinputlisting[caption={sdh\_with\_connector.xacro}, label=lst:urdf_sdh]{code/sdh_with_connector.xacro}

\section{Torso model}
\lstinputlisting[caption={torso.xacro}, label=lst:urdf_torso]{code/torso.xacro}

\section{Table model}
\lstinputlisting[caption={table.xacro}, label=lst:urdf_table]{code/table.xacro}

\end{document}

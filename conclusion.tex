%!TEX root = thesis.tex

\chapter{Conclusion}

During the scope of this bachelor project, a custom simulation solution for the IIS lab robot setup was designed, utilizing the V-Rep simulation platform. It was shown, how the simulation scene was assembled and how dynamic models of required robot components are created. The modelling process was shown on the example of the Schunk SDH2 gripper component. The resulting reference scene reflects the current setup in the IIS lab, but can easily be adjusted to reflect other configurations. The ROS interface is not tied to a specific scene but to known components. Therefore it is easy to create additional scenes with different setups.

The corresponding ROS control interfaces for the robot components were implemented as a V-Rep simulator plugin. The plugin is loaded on V-Rep startup and identifies known components within simulation scenes. As soon as such a component is identified, it can be controlled by the proper ROS control interface. The plugin design allows to introduce additional components and extend it's functionality. Allows to visualize accidental collisions.

The second objective of that project was to integrate the motion planning framework MoveIt into the current setup. Therefore a URDF representation of the robot setup was created that exactly describes the involved robot components and their placement within the planning environment. Based on that URDF description, the planning framework was set up and configured. The integration process also required the implementation of an additional ROS node that allows the proper execution of time-parametrized trajectories. That node was designed to communicate with the existing ROS control interface and can be configured to interact with the simulator and the real robot as well. 

Finally, the proper functionality of the planning tools and the simulator was shown on a reference \emph{pick and place} task that can be executed on the simulator and the real robot as well.
